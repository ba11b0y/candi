%%%%%%%%%%%%%%%%%%%%%%%%%%%%%%%%%%%%%%%%%%%%%%%%%%%%%%%%%%%%%%%%%%%%%%%%%%%%%%%%
% Template for USENIX papers.
%
% History:
%
% - TEMPLATE for Usenix papers, specifically to meet requirements of
%   USENIX '05. originally a template for producing IEEE-format
%   articles using LaTeX. written by Matthew Ward, CS Department,
%   Worcester Polytechnic Institute. adapted by David Beazley for his
%   excellent SWIG paper in Proceedings, Tcl 96. turned into a
%   smartass generic template by De Clarke, with thanks to both the
%   above pioneers. Use at your own risk. Complaints to /dev/null.
%   Make it two column with no page numbering, default is 10 point.
%
% - Munged by Fred Douglis <douglis@research.att.com> 10/97 to
%   separate the .sty file from the LaTeX source template, so that
%   people can more easily include the .sty file into an existing
%   document. Also changed to more closely follow the style guidelines
%   as represented by the Word sample file.
%
% - Note that since 2010, USENIX does not require endnotes. If you
%   want foot of page notes, don't include the endnotes package in the
%   usepackage command, below.
% - This version uses the latex2e styles, not the very ancient 2.09
%   stuff.
%
% - Updated July 2018: Text block size changed from 6.5" to 7"
%
% - Updated Dec 2018 for ATC'19:
%
%   * Revised text to pass HotCRP's auto-formatting check, with
%     hotcrp.settings.submission_form.body_font_size=10pt, and
%     hotcrp.settings.submission_form.line_height=12pt
%
%   * Switched from \endnote-s to \footnote-s to match Usenix's policy.
%
%   * \section* => \begin{abstract} ... \end{abstract}
%
%   * Make template self-contained in terms of bibtex entires, to allow
%     this file to be compiled. (And changing refs style to 'plain'.)
%
%   * Make template self-contained in terms of figures, to
%     allow this file to be compiled. 
%
%   * Added packages for hyperref, embedding fonts, and improving
%     appearance.
%   
%   * Removed outdated text.
%
%%%%%%%%%%%%%%%%%%%%%%%%%%%%%%%%%%%%%%%%%%%%%%%%%%%%%%%%%%%%%%%%%%%%%%%%%%%%%%%%

\documentclass[letterpaper,twocolumn,10pt]{article}
\usepackage{usenix}

% to be able to draw some self-contained figs
\usepackage{tikz}
\usepackage{amsmath}

% Use external bibliography file

%-------------------------------------------------------------------------------
\begin{document}
%-------------------------------------------------------------------------------

%don't want date printed
\date{}

% make title bold and 14 pt font (Latex default is non-bold, 16 pt)
\title{\Large \bf candi: Collusion ANomaly Detection using Interpretability}

%for single author (just remove % characters)
\author{
{\rm Harith Laxman}\\
Virginia Tech
\and
{\rm Rahul Tiwari}\\
Virginia Tech
% copy the following lines to add more authors
% \and
% {\rm Name}\\
%Name Institution
} % end author

\maketitle

%-------------------------------------------------------------------------------
\begin{abstract}
%-------------------------------------------------------------------------------
With AI agents gaining popularity, there is a growing concern about the ability of AI agents to circumvent security oversight through secret collusion. Recent work shows that colluding models can embed covert communication strategies undetectable to overseers, and these capabilities grow with model size. While existing detection methods such as monitoring, paraphrasing, and parameter optimization are not effective, mechanistic anomaly detection (MAD) offers a promising approach. We propose to use mechanistic anomaly detection and interpretability to detect secret collusion and build a comprehensive evaluation framework for multi-agent settings. Our approach leverages patterns in neural network activations to track true model behavior even when outputs are colluding, providing a principled path toward detecting and preventing collusion that bypasses traditional defenses.
\end{abstract}


%-------------------------------------------------------------------------------
\section{Problem Definition}
%-------------------------------------------------------------------------------

With AI agents gaining popularity, there is a growing concern about the ability of AI agents to circumvent security oversight.
Recent work \cite{motwani2024secret} shows that colluding models can embed covert communication strategies undetectable to overseers.
The work also ascertained that as models grow larger, the capabilities of the colluding models to perform secret collusion also grow.
While these findings establish the problem, the work does not provide a strong methodology to detect secret collusion.
Existing methods to detect secret collusion such as monitoring, paraphrasing, and parameter optimization, are not effective.
While mechanistic anomaly detection (MAD) is a promising approach to detect secret collusion, it has mostly been used to detect backdoors \cite{jenner2024gentle}.

%-------------------------------------------------------------------------------
\section{Proposed Method}
%-------------------------------------------------------------------------------

Mallen et al. \cite{mallen2023eliciting} in their work have shown that patterns in neural network's activations can track the true
behavior of the model even when the model's output is hard to verify or in our case colluding.
We propose to use mechanistic anomaly detection (MAD) and interpretability in general to detect secret collusion
and build a comprehensive evaluation framework to detect secret collusion in multi-agent settings.

%-------------------------------------------------------------------------------
\section{Timeline}
%-------------------------------------------------------------------------------

\begin{itemize}
    \item Week 1: Reproduce the results of the paper \cite{motwani2024secret} in a multi-agent setting with open-source models and collect conversation transcripts of the colluding models.
    \item Week 2: Set up instrumentation to collect internal activations and attention patterns in both colluding and regular scenarios.
    \item Week 3: Apply mechanistic anomaly detection (clustering, PCA, probing) to find distinguishing features of collusion.
    \item Week 4: Evaluate the proposed method and setup a comprehensive evaluation framework.
    \item Week 5: Draft all findings in a report.
\end{itemize}


\bibliographystyle{plain}
\bibliography{ref}

%%%%%%%%%%%%%%%%%%%%%%%%%%%%%%%%%%%%%%%%%%%%%%%%%%%%%%%%%%%%%%%%%%%%%%%%%%%%%%%%
\end{document}
%%%%%%%%%%%%%%%%%%%%%%%%%%%%%%%%%%%%%%%%%%%%%%%%%%%%%%%%%%%%%%%%%%%%%%%%%%%%%%%%

%%  LocalWords:  endnotes includegraphics fread ptr nobj noindent
%%  LocalWords:  pdflatex acks
